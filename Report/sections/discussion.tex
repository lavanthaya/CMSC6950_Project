\section{Discussion}

The field of scientific computing consists of a more complex computation process. Therefore, to facilitate it’s process for easy management and code development, it is necessary to use a supportive computation tool to make life easier. For example, bash script automation, make file, git, Github, data storage models, cloud and parallel computing are some of those tools and technologies that exist to help us work on computation models and develop solutions in more manageable way. In this project, some of those tools were used to demonstrate the learned fluency and to present the ability to work with those tools. \newline 

\noindent The project's base is the Python programming language, which is becoming trending worldwide and is the most potential and faster-growing language in the last five years[1]. It offers modern and powerful frameworks for almost all fields. Moreover, as it’s open-source, can always get community support and can see continuous growth. As argoPy also a python based software or library, it’s more convenient to work on Arogo dataset in a python program. Python works great with data and data processing. Therefore, it’s a wise decision that argo API was developed in python. \newline

\noindent ArgoPy provides the most convenient API for fetching Argo data through a simple call. By default, it uses the xarray data model. It’s an open-source Python package that helps to work easily with labelled multi-dimensional arrays. Still, it was transformed into CSV for this project's tasks as it gives a clear view of the data for calculation and visualization. Argo was developed to provide Argo data to experts and non-expert users in ocean science. So it has an option for selecting user modes as either standard or expert. Standard users can just focus on the measurements of ocean data for scientific analysis, and no need to bother about Argo's multitude of variables and parameters. The default user-mode is standard, and this entire project data was handled in standard user mode. \newline

\noindent In this project, Github was used as a version control tool and to manage the source code and for the development, VSCode IDE was used. Several packages need to be installed to set up the environment, and it was partially automated by using a bash script. Finally, to compile and execute the whole project, a makefile script was created to orchestrate the job. So all these clearly show how computation tools help in scientific computing or any other field involving computation. \newline
\newpage
