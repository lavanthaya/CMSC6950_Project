\section{Introduction}

In our daily routine, we are using computation in different areas, especially in scientific research, computation plays an important role. There are many tools to facilitate the computation needs, like writing a program to perform a task, automating the job, maintaining different versions of source code,  shared platform to work on the same project, different data models for storing data collection to be able to retrieve easily and many more. In this project, we used a few of those tools to implement a program to solve two computation tasks and to visualize the outcome.\newline 

\noindent ArgoPy is a python library for Argo data analysis. Argo is a real-time global ocean observation system that provides thousands of highly accurate data of ocean measurements on a daily basis. There are many people who work on oceanography but might have good computation fluency or not at all. ArgoPy helps all levels of people to access Argo ocean data and process it as needed. It’s developed in python, which is currently one of the most trending programming language and becoming widely used by the scientific community. Argo collects data from floats that were deployed in the ocean all over the world. Data from the floats are collected by satellite in real-time, processed, merged in a single dataset and made available as open-source to anyone through an ftp server or monthly zip snapshots \cite{argopy}.\newline

\noindent For this project's scope, a python program was developed to perform two computational tasks on Argo dataset and visualize it. For the first task, the program loads the data, converts it into the dataframe, performs some data manupilation with columns, and saves it into a csv file. The visualization program loads that CSV file and plots the temperature data on latitude, longitude, and depth axis for each month. The second task was to load the temperature, pressure and salinity data and calculate the speed of sound under the ocean at different depths. The calculation results were plotted in 2D graph along with depth.\newline

\noindent For the development, Linux OS was used with python version 3.8. Anaconda is a tool for creating a virtual environment, and it also works as a package manager. VScode was used as a text editor or IDE by connecting it to the Linux machine via ssh. All the dependencies were installed inside the virtual environment, and the packages were installed with conda and pip (python package manager). A bash script was written to automate the deployment so that the dependencies doesn’t need to be installed manually. Throughout the development phase, source code was managed in GitHub. A new repository was created in GitHub and committed and pushed the updates regularly. To maintain a clean development, task one was done in one branch and task two was done in another branch, and finally, both were merged into master.\newline
\newpage
