\section{Methodology}
\subsection{Task-01}

This project includes two major tasks, each with one computation and one visualization of computation results. The first task was implemented to load the dataset from Argo by using the Argopy python API and manipulate the data. As a part of the computation, the format of the dataset was changed. Initially, it was loaded as a xarray in netCDF data model and then it was changed to dataframe as it will be easier to handle the data in two dimensional. Basically, xarray holds the data in multidimensional, and it has attributes for each data that would be useful to understand the dataset. But when converting it to dataframe those attributes will be lost. It was observed that the columns didn’t have a meaningful name, and there were many unwanted columns to process. So as a part of the data pre-processing or cleaning, those unnecessary columns were deleted and changed the remaining column names to be more descriptive. Finally, the modified dataframe was saved in the local datastore in CSV format. \newline 

\noindent For the scope of the visualization of this computation and data pre-processing, the ocean temperature variable was used along with the depth and geographical coordinate space. For the plotting, data was loaded from a csv file created and saved from previous computation tasks. The given dataset doesn’t contain a field for depth, but it has pressure data in decibar units. In ocean science, a unit of decibar of pressure can be taken as a depth in meters. A 3D scatter model was plotted with longitude and latitude on the X and Y axis and the depth on Z-axis. The scatter points were coloured according to it’s positional temperature values. The ocean temperature variation can be observed based on coordinates, ocean depth, and month of the year. In above plot, a single visualization represents a data for one month. Multiple plots were created for each month from the loaded dataset and saved to visualize the temperature variation throughout the month. To produce a final output of visualization, another function was created to combine monthly temperature plots and make a single gif file showing an animated flow of each plot.

\newpage
\subsection{Task-02}

Argo dataset contains many ocean variables such as ocean temperature, salinity and pressure. One of the ocean variables that depend on those parameters is sound speed in the ocean. In general, we know the speed of sound in the air. But when it comes to the ocean, the speed of sound will change based on temperature, salinity and depth. It’s one of the interested parameters to consider in ocean science. While using sensors that works based on the sound wave (Eg: sonar), it’s important to know the speed of sound at that area of the ocean. Therefore, the second task was focused on calculating the speed of sound. \newline

\noindent There are many equations and models that exist for sound speed calculation under the ocean. For this task, the international standard algorithm, often known as the UNESCO algorithm, was used to calculate the speed of sound \cite{xkcd}. The algorithm was programmed in a sound speed calculation script as a function. It takes temperature (units celsius), salinity (units parts per thousand) and pressure (bar) as input arguments. The pressure value in the dataset is in decibar unit, so it was converted to bar before feeding the data into the sound speed calculation function. Finally, the function output was added as an additional column into the data frame and saved as a CSV file. \newline

\noindent Following the above calculation computation process, the visualization script was developed to plot the output of the sound speed calculation. It loads the CSV file created from the previous computation script and loads the data for visualization. The visualization idiom was designed to plot depth Vs sound speed with a third parameter as temperature in one plot and salinity in another plot, so two types of sound speed plots it produces. By following the same method in task one visualization, it also generated multiple plots for each month, and finally, it combines all together into one GIF animation file that shows the sound speed visualization for each month. In such a way, finally, two GIF file were created for depth Vs sound speed with temperature and salinity throughout the year. 

\newpage

